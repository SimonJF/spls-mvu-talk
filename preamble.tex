\usepackage{amsmath, amssymb, stmaryrd, mathpartir, commath, url, longtable,hyperref,tabularx}

\usepackage{forloop, microtype}
\usepackage{wasysym}
%\usepackage[dvipsnames]{xcolor}
\usepackage{csquotes}
%\usepackage{mathptmx}
\usepackage{listings}
\hypersetup{colorlinks=true,citecolor=blue}
\RequirePackage{subcaption}
\captionsetup{compatibility=false}
\usepackage{graphicx}
\usepackage{xspace}
\usepackage{dirtytalk}
\usepackage{xparse}% http://ctan.org/pkg/xparse
\usepackage{etoolbox}% http://ctan.org/pkg/etoolbox
\newcommand{\lambdacalc}{\ensuremath{\lambda}-calculus\xspace}
\newenvironment{fake}[1]{\par\vspace{3pt}\noindent\textbf{#1}\itshape}{\normalfont\ignorespacesafterend\vspace{3pt}\par}
\newcommand{\calcwd}[1]{\textbf{\textsf{#1}}}
\newcommand{\one}{\ensuremath{\mathbf{1}}}
\newcommand{\mvu}{\ensuremath{\lambda_{\mathsf{MVU}}}\xspace}
\newcommand{\mvupi}{\ensuremath{\lambda_{\mathsf{MVU}}^{\pi}}\xspace}
\newcommand{\app}{\:}
\newcommand{\oftype}{\: {:} \:}
\newcommand{\letin}[3]{\calcwd{let} \: #1 = #2 \: \calcwd{in} \: #3}
\newcommand{\letintwo}[2]{\calcwd{let} \: #1 = #2 \: \calcwd{in}}

\newcommand{\caseof}[2]{\calcwd{case} \: #1 \: \{ #2 \}}
\newcommand{\caseofone}[1]{\calcwd{case} \: #1 \: }
\newcommand{\midspace}{\: \mid \:}
\newcommand{\mkwd}[1]{\ensuremath{\mathsf{#1}}}
\newcommand{\antiquote}[1]{\mathbf{\{} #1 \mathbf{\}}}
\newcommand{\htmlty}[1]{\mkwd{Html}(#1)}
\newcommand{\attrty}[1]{\mkwd{Attr}(#1)}
\newcommand{\tagname}[1]{\textsf{#1}}
\newcommand{\smalllt}{\scalebox{0.85}{<}}
\newcommand{\smallgt}{\scalebox{0.85}{>}}
\newcommand{\htmltag}[3]{\smalllt \tagname{#1} \: #2 \smallgt #3 \smalllt / \tagname{#1} \smallgt}
\newcommand{\htmltagzero}[2]{\smalllt \tagname{#1} \smallgt #2 \smalllt / \tagname{#1} \smallgt}
\newcommand{\htmltagqueue}[3]{\smalllt \tagname{#1} \qsep #2 \smallgt #3 \smalllt / \tagname{#1} \smallgt}
\newcommand{\tagzero}[1]{\smalllt \tagname{#1}\smallgt}
\newcommand{\tagzeroend}[1]{\smalllt / \tagname{#1}\smallgt}
\newcommand{\opentag}[1]{\smalllt \tagname{#1}}
\newcommand{\closetag}{\smallgt}

\newcommand{\config}[1]{\mathcal{#1}}
\newcommand{\htmlterm}[1]{\calcwd{html} \app #1}
\newcommand{\attrterm}[1]{\calcwd{attribute} \app #1}
\newcommand{\seq}[1]{\overrightarrow{#1}}
\newcommand{\inl}[1]{\calcwd{inl} \: #1}
\newcommand{\inr}[1]{\calcwd{inr} \: #1}
\newcommand{\evtty}[1]{\ensuremath{\mkwd{ty}(#1)}}
\newcommand{\evt}[1]{\ensuremath{\mkwd{#1}}}
\newcommand{\qstr}[1]{\texttt{"}#1\texttt{"}}

\newcommand{\ev}{\evt{ev}\xspace}
\newcommand{\evtpayload}[2]{\evt{#1}(#2)}
\newcommand{\clickevt}{\ensuremath{\attribute{click}}}
\newcommand{\inputevt}{\ensuremath{\attribute{input}}}
\newcommand{\keydownevt}{\ensuremath{\attribute{keyDown}}}
\newcommand{\keyupevt}{\ensuremath{\attribute{keyUp}}\xspace}
\newcommand{\keypressevt}{\ensuremath{\attribute{keyPress}}\xspace}
\newcommand{\mousemoveevt}{\ensuremath{\attribute{mouseMove}}\xspace}
\newcommand{\vh}{D}
\newcommand{\pgctx}[1]{\config{D}[#1]}
\newcommand{\mh}{M\xspace}
\newcommand{\vc}{V_{\textsf{C}}\xspace}
\newcommand{\va}{V_{\textsf{a}}\xspace}

\newcommand{\ctxh}{E_{\textsf{H}}}
\newcommand{\ctxa}{E_{\textsf{a}}}

\newcommand{\state}[2]{(#1, #2)}
\newcommand{\statecomb}[3]{(#1, #2, #3)}
\newcommand{\statesub}[3]{(#1, #2, #3)}
\newcommand{\handlerproc}[3]{\langle #1 \mid #2 \mid #3 \rangle}
\newcommand{\handlerprocexp}[4]{\langle #1 \mid \state{#2}{#3} \mid #4 \rangle}
\newcommand{\handlerprocsub}[5]{\langle #1 \mid \statesub{#2}{#3}{#4} \mid #5 \rangle}
\newcommand{\idle}[1]{\calcwd{idle} \: {#1}}
\newcommand{\sys}[2]{#1 \fatsemi\: #2}
\newcommand{\syssub}[4]{#1 \fatsemi\: #2 \fatsemi\: #3 \fatsemi\: #4}

\newcommand{\evalarrow}{\longrightarrow}
\newcommand{\teval}{\evalarrow_{\textsf{M}}}
\newcommand{\ceval}{\evalarrow}
\newcommand{\cevalminus}{\evalarrow_{\textsf{E}}}
\newcommand{\totheleft}[1]{\begin{flushleft}#1\end{flushleft}}
\newcommand{\evthandler}[1]{\ensuremath{\attribute{#1}}}
\newcommand{\getevthandler}[3]{#1(#2, #3)}
\newcommand{\handler}[1]{\mkwd{handler}(#1)}
\newcommand{\payload}[1]{\mkwd{payload}(#1)}
\newcommand{\id}[1]{\mkwd{id}(#1)}

\newcommand{\handle}[1]{\mkwd{handle}(#1)}

\newcommand{\defeq}{\triangleq}
\newcommand{\vdashs}{\vdash_{\textsf{T}}}
\newcommand{\ttrue}{\mkwd{true}}
\newcommand{\ffalse}{\mkwd{false}}
\newcommand{\intty}{\mkwd{Int}}
\newcommand{\boolty}{\mkwd{Bool}}
\newcommand{\stringty}{\mkwd{String}}

\newcommand{\deriv}[1]{\mathbf{#1}}
\newcommand{\produces}[1]{\:{!}\:#1}
\newcommand{\thread}[1]{(\!( #1 )\!)}
\newcommand{\wcirc}{\circ}
\newcommand{\bcirc}{\bullet}

\newcommand{\subscriptionty}[1]{\mkwd{Sub}(#1)}
\newcommand{\subscription}[2]{\calcwd{sub} \: #1 \: #2}
\newcommand{\subempty}{\calcwd{subEmpty}}
\newcommand{\emptylist}{\textbf{[}~\textbf{]}}
\newcommand{\cons}[2]{#1 :: #2}
\newcommand{\eh}{h}
\newcommand{\listty}[1]{\mkwd{List}(#1)}
\newcommand{\metadef}[1]{\mkwd{#1}}
\newcommand{\events}[1]{\metadef{events}(#1)}
\newcommand{\vs}{{V_{\mathsf{S}}}}
\newcommand{\evthandlers}[2]{\mkwd{eventHandlers}(#1, #2)}
\newcommand{\desugar}[1]{\llbracket #1 \rrbracket}

\newcommand{\gvconst}[1]{\calcwd{#1}}
\newcommand{\gvsend}[2]{\calcwd{send} \app (#1, #2)}
\newcommand{\gvrecv}[1]{\calcwd{receive} \app #1}
\newcommand{\gvcancel}[1]{\calcwd{cancel} \app #1}
\newcommand{\gvclose}[1]{\calcwd{close} \app #1}
\newcommand{\gvnew}[1]{\calcwd{new} \: #1}
\newcommand{\tryasinotherwise}[4]{\calcwd{try} \: #1 \: \calcwd{as} \: #2 \: \calcwd{in} \: #3 \: \calcwd{otherwise} \: #4}
\newcommand{\raiseexn}{\calcwd{raise}\xspace}
\newcommand{\ppos}[1]{#1^{+}}
\newcommand{\pneg}[1]{#1^{-}}
\newcommand{\gvout}[2]{{!}#1.#2}
\newcommand{\gvin}[2]{{?}#1.#2}
\newcommand{\gvoutone}[1]{{!}#1}
\newcommand{\gvinone}[1]{{?}#1}

\newcommand{\gvend}{\mkwd{End}}
\newcommand{\gvqueue}[4]{#1(#2)\leftrightsquigarrow #3(#4)}
\newcommand{\zap}[1]{\lightning #1}
\newcommand{\gvdual}[1]{\overline{#1}}
\newcommand{\without}{/}
\newcommand{\apty}[1]{\mkwd{AP}(#1)}
\newcommand{\var}[1]{\textit{#1}}


\lstdefinelanguage{Links}{%
  morekeywords={typename, fun, linfun, op, var, if, this, true, false, else, case, switch, handle,
    handler, shallowhandler, open, do, sig, new, send, receive, spawnAt, spawn,
module, request, accept, try, as, otherwise, catch, offer, select, raise,
fork, spawnClient, cancel, catch, page, close, Any, Type, Unl, forall, vdom},%
  sensitive=t, %
  comment=[l]{\#\ },%
  escapeinside={(*}{*)},%
  morestring=[d]{"},%
  keywordstyle=\color{blue},
  showstringspaces=false
  %frame = single
 }

% Links style
\lstset{
  basicstyle=\linespread{0.86}\ttfamily\small,
  keywordstyle=\bfseries,
  language=Links,
  backgroundcolor=\color{white}
}


\newcommand{\idattr}{\textit{id}\xspace}
\newcommand{\desugarterm}[1]{\llbracket #1 \rrbracket}
\newcommand{\desugarhtml}[1]{\llbracket #1 \rrbracket}
\newcommand{\desugarattr}[1]{\llbracket #1 \rrbracket}

\newcommand{\coretagone}[1]{\calcwd{htmlTag} \: \tagname{#1}}
\newcommand{\coretagtwo}[2]{\calcwd{htmlTag} \: \tagname{#1} \: #2}
\newcommand{\coretag}[3]{\calcwd{htmlTag} \: \tagname{#1} \: #2 \: #3}
\newcommand{\htmltext}[1]{\calcwd{htmlText} \: #1}
\newcommand{\htmlempty}{\calcwd{htmlEmpty}}
\newcommand{\append}[2]{#1 \star #2}

\newcommand{\attr}[2]{\calcwd{attr} \: #1 \: #2}
\newcommand{\attrempty}{\calcwd{attrEmpty}}
\newcommand{\ak}{\mathit{ak}}
\newcommand{\at}{\mathit{at}}
\newcommand{\run}[3]{\calcwd{run} \: #1 \: #2 \: #3}
\newcommand{\runsub}[4]{\calcwd{run} \: #1 \: #2 \: #3 \: #4}

\lstdefinelanguage{JavaScript}{
  morekeywords=[1]{break, continue, delete, else, for, function, if, in,
    new, return, this, typeof, var, void, while, with},
  % Literals, primitive types, and reference types.
  morekeywords=[2]{false, null, true, boolean, number, undefined,
    Array, Boolean, Date, Math, Number, String, Object, const},
  % Built-ins.
  morekeywords=[3]{eval, parseInt, parseFloat, escape, unescape},
  sensitive,
  morecomment=[s]{/*}{*/},
  morecomment=[l]//,
  morecomment=[s]{/**}{*/}, % JavaDoc style comments
  morestring=[b]',
  morestring=[b]"
}[keywords, comments, strings]

\newcommand{\bl}{\begin{array}{l}}
\newcommand{\el}{\end{array}}
\newcommand{\intstr}[1]{\mkwd{intToString}(#1)}
\newcommand{\cmdty}[1]{\mkwd{Cmd}(#1)}
\newcommand{\cmdempty}{\calcwd{cmdEmpty}}
\newcommand{\cmdspawn}[1]{\calcwd{cmdSpawn} \: #1}
\newcommand{\cmdspawnlinear}[1]{\calcwd{cmdSpawnLinear} \: #1}
\newcommand{\un}[1]{\mkwd{un}(#1)}
\newcommand{\procs}[1]{\mkwd{procs}(#1)}

\definecolor{shade}{RGB}{215,215,215}
\newcommand{\shade}[1]{\setlength{\fboxsep}{0pt}\colorbox{shade}{\vphantom{$\mid$}\ensuremath{#1}}}
%\newcommand{\shade}[1]{#1}
\newcommand{\shadetext}[1]{\setlength{\fboxsep}{0pt}\colorbox{shade}{#1}}
\newcommand{\hlred}[1]{\color{red}{#1}}
% \makeatletter
% \setlength{\@fptop}{0pt}
% \makeatother
\newcommand{\transition}[5]{\calcwd{transition} \: #1 \: #2 \: #3 \: #4 \: #5}
\newcommand{\transitionalone}[3]{\calcwd{transition} \: #1 \: #2 \: #3}
\newcommand{\transitionty}[2]{\mkwd{Transition}(#1, #2)}
\newcommand{\transitiontyone}[1]{\mkwd{Transition}(#1)}
\newcommand{\notransitionalone}[1]{\calcwd{noTransition} \: #1}
\newcommand{\notransition}[2]{\calcwd{noTransition} \: #1 \: #2}
\newcommand{\handlerproctrans}[4]{\handlerproc{#1}{#2}{#3}_{#4}}
\newcommand{\handlerprocexptrans}[5]{\handlerprocexp{#1}{#2}{#3}{#4}_{#5}}
\newcommand{\handlerprocexptc}[6]{\handlerprocsub{#1}{#2}{#3}{#4}{#5}_{#6}}
\newcommand{\vdashtrans}[2]{\vdash^{#1}_{#2}}
\newcommand{\threadtrans}[2]{(\!( #1 )\!)_{#2}}

\newcommand{\updating}[1]{\calcwd{updating} \: #1}
\newcommand{\extracting}[2]{\calcwd{extracting}_{#1} \: #2}
\newcommand{\extractingt}[3]{\calcwd{extractingT}_{#1, #2} \: #3}
\newcommand{\extractingtexp}[5]{\calcwd{extractingT}_{(#1, #2, #3), #4} \: #5}
\newcommand{\rendering}[2]{\calcwd{rendering}_{{#1}} \: #2}
\newcommand{\renderingext}[3]{\calcwd{rendering}_{{#1}, {#2}} \: #3}
\newcommand{\transitioning}[4]{\calcwd{transitioning}_{{#1, #2, #3}} \: #4}
\newcommand{\transitioningextexp}[6]{\calcwd{transitioning}_{{#1, #2, #3, #4, #5}} \: #6}
\newcommand{\transitioningext}[4]{\calcwd{transitioning}_{{#1, #2, #3}} \: #4}
\newcommand{\version}[1]{\mkwd{version}(#1)}

\newlength{\mylength}
\makeatletter
\newcommand{\mycfs}[1]{%
  \normalsize
  \@defaultunits\mylength=#1pt\relax\@nnil
  \edef\@tempa{{\strip@pt\mylength}}%
  \ifx\protect\@typeset@protect
     \edef\@currsize{\noexpand\mycfs\@tempa}% store calculated size
  \fi
  \mylength=1.2\mylength
  \edef\@tempa{\@tempa{\strip@pt\mylength}}%
  %\@tempa
  \expandafter\fontsize\@tempa
  \selectfont
}
\makeatother

\renewcommand{\footnotesize}{\mycfs{8}}

\newcommand{\pagety}[1]{\mkwd{Page}(#1)}
\newcommand{\pgtag}[4]{\calcwd{htmlTag}_{#4} \: \tagname{#1} \: #2 \: #3}
\newcommand{\handlers}[2]{\mkwd{handlers}(#1, #2)}
\newcommand{\diff}[2]{\mkwd{diff}(#1, #2)}

\newcommand{\redrowskip}{0.5em}
\newcommand{\qsep}{\:@\:}
\newcommand{\lin}{\mathsf{L}}
\newcommand{\unr}{\mathsf{U}}
\newcommand{\kind}{\kappa}
\newcommand{\kto}[1]{\to^{#1}}
\newcommand{\rtv}[1]{#1}
\newcommand{\recty}[2]{\mu #1 . #2}
\newcommand{\constty}[1]{\Sigma(#1)}
\newcommand{\vers}{\iota}
\newcommand{\uto}{\kto{\unr}}
\newcommand{\lto}{\kto{\lin}}
\newcommand{\halt}{\calcwd{halt}}
\newcommand{\runcomb}[5]{\calcwd{run} \: #1 \: #2 \: #3 \: #4 \: #5}
\newcommand{\ep}{E_{\textsf{P}}}
\newcommand{\tp}{T_{\textsf{P}}}
\newcommand{\domh}{D_{\mkwd{H}}}
\newcommand{\erase}[1]{\mkwd{erase}(#1)}

\newcommand{\attribute}[1]{\mkwd{#1}}
\newcommand{\textnode}[1]{\mathtt{#1}}
\newcommand{\fn}[1]{\mkwd{fn}(#1)}
\usepackage{tikz-cd}

\tikzstyle{arrow} = [thick,->]
\tikzstyle{diagnode} = [rectangle, rounded corners, minimum width=4cm, minimum
height=1cm,text centered, text width=4cm, draw=black, anchor=center, fill=red!30]


